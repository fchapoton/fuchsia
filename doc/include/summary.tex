{\bf PROGRAM SUMMARY}

\begin{small}
\noindent

{\em Program Title:}
    \fuchsia{}

{\em Authors:}
    O.~Gituliar and V.~Magerya

{\em Program obtainable from:}
    \url{https://github.com/gituliar/fuchsia/}

{\em Journal Reference:}
    %Leave blank, supplied by Elsevier.

{\em Catalog identifier:}
    %Leave blank, supplied by Elsevier.

{\em Licensing provisions:}
    ISC license

{\em Programming language:}
    \python 2.7

{\em Operating system:}
    \linux

{\em RAM:}
    Dependent upon the input data. Expect hundreds of megabytes.

{\em Keywords:}
    Computer algebra, Feynman integrals, Canonical form of DEs, Moser-reduction

{\em Classification:}
    5 Computer Algebra, 11.1 High Energy Physics and Computing

{\em External routines/libraries:}
    \href{http://www.sagemath.org/}{\sage} 7.0 or higher

{\em Nature of problem:}
    Necessity to calculate multi-loop Feynman integrals is a common problem in Quantum Field Theory.
    In recent years, a method of differential equations became popular to perform this task, mainly due to the progress in preforming integration-by-parts reduction.
    In general, to find solutions for these equations is a very challenging task.
    However, in the case when a canonical form of the equations is known solutions, and hence initial Feynman integrals, can be found in a trivial way \cite{Henn13}.

{\em Solution method:}
    \fuchsia finds a canonical form of differential equations (DEs) for Feynman integrals using method proposed by Roman Lee \cite{Lee15}.
    The method consists of three main steps: fuchsification, normalization, and factorization; and is based on analysis of eigenvalues and eigenvectors of DEs at singular points.

{\em Restrictions:}
    No support for systems with (1) symbolic singular points; (2) eigenvalues of residues other than $n + m \eps$, where $n$ is integer.

{\em Running time:}
    Depends upon the input data, its size and complexity.
    Around an hour in total for the example $74\times74$ matrix with 20 singular points on a PC with a 1.7GHz Intel Core i5 CPU.

\end{small}
