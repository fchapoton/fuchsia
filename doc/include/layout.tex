
% Page layout definitions
% ----------------------
  \newlength{\dinwidth} \newlength{\dinmargin}
  \setlength{\dinwidth}{21.0cm}   %width of A4 paper
  \setlength{\textheight}{23.9cm} %height of text
  \setlength{\textwidth}{15.7cm}  %width of text
% Now calculate the margin as 0.5*(paperwidth-textwidth}
  \setlength{\dinmargin}{\dinwidth}     
  \addtolength{\dinmargin}{-\textwidth} 
  \setlength{\dinmargin}{0.5\dinmargin}
% For twoside prints you often want the oddsidemargin to be larger
% than the evensidemargin because staples or punchholes endup at the
% lefthand side of odd numbered pages 
  \setlength{\oddsidemargin}{1.2\dinmargin}
  \setlength{\evensidemargin}{0.8\dinmargin}
% Latex defines the margin one inch from the paper border 
  \addtolength{\oddsidemargin}{-1in}
  \addtolength{\evensidemargin}{-1in}
%
  \setlength{\marginparwidth}{0.9\dinmargin} \setlength{\itemsep}{0mm}
  \setlength{\parsep}{0mm} \marginparsep 8pt \marginparpush 5pt
  \topmargin -42pt \headheight 12pt \headsep 30pt \footskip 24pt
  \parskip 2mm \parindent 0mm \font\bollox=cmmib10 scaled \magstep4
 