{\bf PROGRAM SUMMARY}

\begin{small}
\noindent

{\em Program Title:}
    \fuchsia

{\em Authors:}
    O.~Gituliar and V.~Magerya

{\em Program obtainable from:}
    \url{https://github.com/gituliar/fuchsia/}

{\em Journal Reference:}
    %Leave blank, supplied by Elsevier.

{\em Catalog identifier:}
    %Leave blank, supplied by Elsevier.

{\em Licensing provisions:}
    ISC license

{\em Programming language:}
    \python 2.7

{\em Operating system:}
    \linux, \texttt{Unix}-like

{\em RAM:}
    Dependent upon the input data. Expect hundreds of megabytes.

{\em Keywords:}
    Computer algebra, Feynman integrals, differential equations, epsilon form, Fuchsian form, Moser reduction

{\em Classification:}
    5 Computer Algebra, 11.1 High Energy Physics and Computing

{\em External routines/libraries:}
    \href{http://www.sagemath.org/}{\sage} (7.0 or higher), \maple (optional)

{\em Nature of problem:}
    Feynman master integrals may be calculated from solutions of a linear system of differential equations with rational coefficients.
    Such a system can be easily solved as an $\eps$-series when its epsilon form is known.
    Hence, a tool which is able to find the epsilon form transformations can be used to evaluate Feynman master integrals. 

{\em Solution method:}
    The solution method is based on the Lee algorithm \cite{Lee15} which consists of three main steps: fuchsification, normalization, and factorization.
    During the fuchsification step a given system of differential equations is transformed into the Fuchsian form with the help of the Moser method \cite{Mos59}.
    Next, during the normalization step the system is transformed to the form where eigenvalues of all residues are proportional to the dimensional regulator $\eps$.
    Finally, the system is factorized to the epsilon form by finding an unknown transformation which satisfies a system of linear equations.

{\em Restrictions:}
    Systems of single-variable differential equations are considered.
    A system needs to be reducible to Fuchsian form and eigenvalues of its residues must be of the form $n + m\,\eps$, where $n$ is integer.

{\em Running time:}
    Around an hour in total for an example $74\times74$ matrix with 20 singular points on a PC with a 1.7GHz Intel Core i5 CPU, but dependent upon input matrix, its size, number of singular points and their degrees.
    An additional slowdown is to be expected for matrices with complex and/or irrational singular point locations, as these are particularly difficult for symbolic algebra software to handle.

\end{small}
