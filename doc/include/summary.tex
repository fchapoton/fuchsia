{\bf PROGRAM SUMMARY}

\begin{small}
\noindent

{\em Program Title:}
    \fuchsia

{\em Authors:}
    O.~Gituliar and V.~Magerya

{\em Program obtainable from:}
    \url{https://github.com/gituliar/fuchsia/}

{\em Journal Reference:}
    %Leave blank, supplied by Elsevier.

{\em Catalog identifier:}
    %Leave blank, supplied by Elsevier.

{\em Licensing provisions:}
    ISC license

{\em Programming language:}
    \python 2.7

{\em Operating system:}
    \linux, \texttt{Unix}-like

{\em RAM:}
    Dependent upon the input data. Expect hundreds of megabytes.

{\em Keywords:}
    Computer algebra, Feynman integrals, differential equations, Canonical form, Fuchsian form, Moser reduction

{\em Classification:}
    5 Computer Algebra, 11.1 High Energy Physics and Computing

{\em External routines/libraries:}
    \href{http://www.sagemath.org/}{\sage} 7.0 or higher

{\em Nature of problem:}
    Feynman master integrals may be calculated from solutions of a linear system of differential equations with rational coefficients.
    Such a system can be easily solved as an $\eps$-series when its epsilon form is known.
    Hence, a tool which is able to find the epsilon form transformations can be used to evaluate Feynman master integrals. 

{\em Solution method:}
    Methods of J.Moser \cite{Mos59} and R.N.Lee \cite{Lee15}.

{\em Restrictions:}
    Differential equations in one variable only are considered.
    A system needs to be reducible to Fuchsian form and eigenvalues of its residues must be of the form $n + m \eps$, where $n$ is integer.

{\em Running time:}
    Depends upon the input data, its size and complexity.
    Around an hour in total for an example $74\times74$ matrix with 20 singular points on a PC with a 1.7GHz Intel Core i5 CPU.

\end{small}
